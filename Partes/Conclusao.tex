\chapter{\uppercase{Considerações finais}}
\label{conclusao}

No começo da disciplina quando nos foi posto a realização de um desenvolvimento de um chatbot para auxiliar as pessoas a sanar dúvidas sobre segurança na internet. Após a proposta, foi necessário fazer várias pesquisas sobre funcionamento de IAs, de BOTs e como são feitos, além de fazer pesquisas para saber da população quais são suas maiores dúvidas para sanar as mesmas e auxiliar cada vez mais as pessoas, de forma gratuita e de fácil acesso.
 
 O desenvolvimento do bot educativo ``Anna" levou em sua totalidade, quase 5 meses, foi e está sendo uma grande experiênca enriquecedora.
 
 Com as informações passadas ao longo da apresentação do Projeto Integrador, é conclusivo que os objetivos apresentados pelo P.O foram alcançados, porém é possível implementar o bot continuamente, melhorando-o.
 
  \section{Trabalhos Futuros}
  
  Este chatbot foi testado em sala, com pessoas fora de sala também, a capacidade de crescimento do bot é de certa forma conhecido, sua versão grátis é possível fazer até 50 interações. A Eficácia e confiabilidade estão diretamente ligadas ao 