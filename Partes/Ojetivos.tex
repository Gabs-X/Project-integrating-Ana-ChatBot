\section{Objetivos}
\label{objetivos}

%Assumindo que existe um problema a ser resolvido, apresente qual o objetivo de seu projeto de pesquisa. O que você pretende (ou pretendeu) exatamente fazer. Aqui, deve aparecer a principal ``contribuição'' de seu projeto. Qual é a principal ``coisa'' que você pretende/pretendeu fazer? Qual sua principal entrega? Não é necessário criar uma subseção para cada tipo. Pode haver uma única seção, chamada de ``objetivos'' cujo texto divida-se naturalmente em objetivo geral e objetivos específicos, deixando claro qual caso está sendo tratado em cada momento. Para diferenciar o objetivo geral dos objetivos específicos, siga as seguintes diretrizes:

\begin{itemize}

\item \textbf{Objetivo geral}: O objetivo é desenvolver uma ferramenta de conversação, via um chatbot, com processamento de linguagem natural para agilizar o atendimento. 
		
\item \textbf{Objetivos específicos}: Pesquisa bibliográfica sobre os assuntos de chatbot e helpdesk.
Realizar pesquisa dos trabalhos correlatos com este Projeto.
Identificar o público-alvo do projeto.
Definir as atividades de cada integrante do projeto.
Pesquisar sobre como construir e funcionamento de Bots e escolher uma plataforma
para desenvolvimento.
Definir os assuntos e tópicos sobre segurança da informação para aplicar no Bot.
Construção de um protótipo do Bot.
Aplicar testes no protótipo.
Analisar os resultados.
Apresentar os resultados na mostra de Extensão.
\end{itemize}
